\documentclass{scrartcl}
\usepackage{amsmath}
\usepackage{amssymb}
\usepackage{graphicx}
\usepackage{epstopdf}
\usepackage{geometry}
\usepackage[T1]{fontenc}
\usepackage[polish]{babel}
\usepackage[utf8]{inputenc}
\geometry{left=2.5cm,right=2.5cm,top=2.5cm,bottom=2.5cm}

\begin{document}

\title{Implementacja i badanie skuteczności algorytmu k najbliższych sąsiadów (k-NN)}

\subtitle{Konspekt}

\author{Kinga Pilch, Hubert Rosiak}

\maketitle

\section*{Krótki opis algorytmu}

Algorytm k najbliższych sąsiadów (ang. \textit{k nearest neighbours}) jest prostym algorytmem służącym do rozwiązywania problemu klasyfikacji lub regresji. Idea algorytmu jest następująca: mając zbiór danych uczących w pewnej przestrzeni, dla nowej, nieznanej wcześniej obserwacji prognozujemy wartość będącą średnią arytmetyczną wartości jej k najbliższych (według pewnej metryki, najczęściej euklidesowej) sąsiadów ze zbioru uczącego. Gdy mamy do czynienia z problemem klasyfikacji, nowej obserwacji przypisujemy klasę najliczniej występującą wśród jej k najbliższych sąsiadów.

Algorytm k najbliższych sąsiadów można łatwo modyfikować. Wybór parametru k może w znaczący sposób wpływać na zachowanie algorytmu. Podobnie zamiast średniej arytmetycznej można zastosować inny sposób wyboru nowej wartości/klasy na podstawie informacji uzyskanych od sąsiadów - jedną z popularnych modyfikacji tego rodzaju jest zastosowanie średniej ważonej. Używana w algorytmie metryka służąca obliczaniu odległości między obserwacjami również może podlegać zmianie.

\section*{Wybrany język programowania}

Projekt stworzony zostanie w języku Python.

\begin{thebibliography}{1}
\bibitem{knn_weighted} 
Sahibsingh A. Dudani, 
\textit{The Distance-Weighted k-Nearest-Neighbour Rule},
IEEE Transactions on Systems, Man, and Cybernetics (Vol: SMC-6, Issue:4, April 1976)
\end{thebibliography}

\end{document}